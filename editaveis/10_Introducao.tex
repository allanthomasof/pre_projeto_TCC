%\chapter*{Introdução}
%\begin{center}
\vspace{1.2em}
\textbf{\large INTRODUÇÃO}

\vspace{2.9em}
%\end{center}
%\thispagestyle{empty} // Essa linha tira a numeração da página

\addcontentsline{toc}{chapter}{INTRODUÇÃO}
\stepcounter{chapter} %incrementa o número do capítulo

	\par No mundo em que vivemos, com o desenvolvimento tecnológico e os \textit{smartphones} em ascensão, as pessoas sempre olham seu dispositivo como recurso para resolver um problema ou como ele pode facilitar uma determinada tarefa, como uma transação bancária, uma pesquisa em um site de buscas ou simplesmente para acessar uma rede social. Esses dispositivos, que estão na mão de praticamente todos os universitários, podem ser usados como grandes aliados de uma instituição de ensino. Hoje em dia, o desenvolvimento e a utilização de \textit{softwares} não se limita apenas aos \textit{desktops} como era há alguns anos atrás, cada vez mais os sistemas seguem atrelados aos dispositivos móveis, com toda a facilidade e mobilidade que eles nos proporcionam.
	
	\par Diante deste cenário, o projeto consiste em criar uma aplicação que possibilite à universidade coletar opiniões dos alunos em relação à sua infraestrutura, professores, instalações, entre outros, algo que já é feito semestralmente através do portal do aluno, mas que pode ser aprimorado com o uso de um aplicativo destinado especialmente para esse objetivo. Uma pesquisa que seria realizada através do portal da universidade, passa a ser feita por um aplicativo móvel e torna ainda mais atrativa para quem possa vir a respondê-la. 

	\par A tecnologia que será usada no desenvolvimento do projeto, que é chamada de híbrida, permite que um único aplicativo possua suporte para mais de uma plataforma \textit{mobile}, como por exemplo o \textit{Android} e o \textit{iOS}, tudo isso a partir do \textit{framework} Ionic. De acordo com \citeonline{khanna2016getting}, aplicativos híbridos são semelhantes aos aplicativos nativos, eles são desenvolvidos com um único código base e utilizados em mais de uma plataforma, além possuírem comunicação com o \textit{hardware} e também serem instalados no dispositivo. Segundo \citeonline{ionicdesc} o Ionic é um poderoso SDK\footnote{\textit{Software Development Kit}} HTML5 que ajuda a construir aplicativos móveis usando tecnologias web como HTML\footnote{\textit{HyperText Markup Language}}, CSS\footnote{\textit{Cascading Style Sheets}} e Javascript, que são voltados principalmente para a interface gráfica do aplicativo.

\chapter{JUSTIFICATIVA}

	\par A escolha por desenvolver um aplicativo usando Ionic e seus conceitos, deve-se pela sua excelente forma de construir um App\footnote{Abreviação para a palavra \textit{Application}} \textit{mobile} multiplataforma, além das tecnologias usadas por ele que estão mais robustas e em constantes atualizações. A principal plataforma utilizada será o \textit{Android} devido à sua grande popularidade e por ser um dos sistemas operacionais mais usados mundialmente.
	
	\par A relevância do aplicativo para a universidade é grande, pois facilitará a pesquisa feita pela CPA, a qual muitos alunos deixam de responder por falta da praticidade. Um aplicativo destinado especialmente para ela, com fácil utilização, intuitivo e que emite alertas aos usuários quando a pesquisa estiver disponível. Pode ajudar a pesquisa a atingir seu público por mais de um meio, aumentado a quantidade de entrevistados, o que, no final, garante ainda mais informações.
	
	\par O trabalho também auxiliará os alunos que se interessarem por aprender como funciona o desenvolvimento de um aplicativo híbrido, além de todas as tecnologias e \textit{frameworks} que estão envolvidos no processo, que são atuais e muito requisitadas no mercado de trabalho.
	

\chapter{OBJETIVOS}

	\par Descreveremos a seguir os objetivos pretendidos pela presente pesquisa.

\section{Objetivo Geral}

	\par Desenvolver um aplicativo móvel multiplataforma a partir do \textit{framework} Ionic, que deve possibilitar aos alunos da UNIVÁS\footnote{Universidade do Vale do Sapucaí} responderem ao questionário semestral que é disponibilizado pela CPA\footnote{Comissão Própria de Avaliação} afim de coletar informações e opiniões referentes à universidade.

\section{Objetivos Específicos}

	\par Afim de atingir o objetivo geral da pesquisa, estão descritos a seguir os objetivos específicos:
	
	\begin{itemize}
		\item Efetuar o levantamento de requisitos do sistema, bem como das tecnologias que serão usadas no desenvolvimento;
		\item Planejar o desenvolvimento do projeto a partir dos requisitos coletados;
		\item Desenvolver o aplicativo que permita aos alunos responderem o questionário da CPA.
	\end{itemize}
	\par Conforme esses objetivos, espera-se demonstrar de maneira simples e efetiva o funcionamento do Ionic e seu método de desenvolvimento híbrido, tirando o maior proveito possível com um software que auxiliará a pesquisa que é realizada pela CPA.

