%\chapter*{Introdução}
%\begin{center}
\vspace{1.2em}
\textbf{\large INTRODUÇÃO}

\vspace{2.9em}
%\end{center}
\thispagestyle{empty}

\addcontentsline{toc}{chapter}{INTRODUÇÃO}
\stepcounter{chapter} %incrementa o número do capítulo

\par No mundo em que vivemos atualmente com o desenvolvimento tecnológico e os \textit{smartphones} em ascensão, as pessoas sempre olham seu dispositivo como principal recurso para resolver todo e qualquer problema, ou algo que possa facilitar determinada tarefa, como uma transação bancária, uma pesquisa em um site de buscas ou simplesmente para acessar uma rede social.
\par Esses dispositivos que estão na mão de praticamente todos os universitários, podem ser usados como grandes aliados da universidade junto a seus alunos.
\par A tecnologia que será usada no desenvolvimento, que é chamada de híbrida, permite que um único aplicativo possua suporte para mais de uma plataforma, como por exemplo o \textit{Android} e o \textit{iOS}, tudo isso a partir do \textit{framework} IONIC.
\par Segundo \citeonline{ionicdesc} o IONIC é um SDK HTML5 poderoso que ajuda a construir aplicativos móveis usando tecnologias web como HTML, CSS e Javascript, que está voltado principalmente para a aparência e interface gráfica do aplicativo.
\par Uma pesquisa que seria realizada através do portal da universidade, que passa a ser feita por um aplicativo, se torna ainda mais atrativa para quem possa vir a respondê-la, o que no final trará ainda mais informações.
\par A ideia consiste em criar uma aplicação que possibilite à universidade, coletar opiniões dos alunos em relação à infraestrutura da mesma e seus colaboradores, algo que já é feito semestralmente através do portal do aluno, mas que pode ser aprimorado com o uso de um aplicativo destinado especialmente para esse objetivo.


\chapter{OBJETIVOS}


\section{Objetivo Geral}


\section{Objetivos Específicos}

\chapter{JUSTIFICATIVAS}

