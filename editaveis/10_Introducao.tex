%\chapter*{Introdução}
%\begin{center}
\vspace{1.2em}
\textbf{\large INTRODUÇÃO}

\vspace{2.9em}
%\end{center}
\thispagestyle{empty}

\addcontentsline{toc}{chapter}{INTRODUÇÃO}
\stepcounter{chapter} %incrementa o número do capítulo

	\par No mundo em que vivemos atualmente com o desenvolvimento tecnológico e os \textit{smartphones} em ascensão, as pessoas sempre olham seu dispositivo como principal recurso para resolver todo e qualquer problema, ou algo que possa facilitar determinada tarefa, como uma transação bancária, uma pesquisa em um site de buscas ou simplesmente para acessar uma rede social. Esses dispositivos que estão na mão de praticamente todos os universitários, podem ser usados como grandes aliados da universidade junto a seus alunos. Hoje em dia o desenvolvimento e a utilização de \textit{softwares} não se limita apenas aos \textit{desktops}, cada vez mais eles seguem atrelados aos dispositivos móveis, com toda a facilidade e mobilidade que eles nos proporcionam.

	\par A tecnologia que será usada no desenvolvimento, que é chamada de híbrida, permite que um único aplicativo possua suporte para mais de uma plataforma, como por exemplo o \textit{Android} e o \textit{iOS}, tudo isso a partir do \textit{framework} Ionic. Segundo \citeonline{ionicdesc} o Ionic é um poderoso SDK HTML5 que ajuda a construir aplicativos móveis usando tecnologias \textit{web} como HTML, CSS e Javascript, que está voltado principalmente para a aparência e interface gráfica do aplicativo.

	\par O projeto consiste em criar uma aplicação que possibilite à universidade, coletar opiniões dos alunos em relação à infraestrutura da mesma e seus colaboradores, algo que já é feito semestralmente através do portal do aluno, mas que pode ser aprimorado com o uso de um aplicativo destinado especialmente para esse objetivo. Uma pesquisa que seria realizada através do portal da universidade, que passa a ser feita por um aplicativo, se torna ainda mais atrativa para quem possa vir a respondê-la, o que no final trará ainda mais informações.

\chapter{OBJETIVOS}

	\par Descreveremos à seguir os objetivos pretendidos com base na presente pesquisa.

\section{Objetivo Geral}

	\par Desenvolver um aplicativo móvel multiplataforma a partir do \textit{framework} Ionic, que deve possibilitar aos alunos da UNIVÁS responderem ao questionário semestral que é disponibilizado pela CPA afim de coletar informações e opiniões referentes à universidade.

\section{Objetivos Específicos}

	\par Afim de atingir o objetivo geral da pesquisa, estão descritos a seguir alguns objetivos específicos.
	\begin{itemize}
		\item Demonstrar o funcionamento do Ionic, suas principais funcionalidades e vantagens.
		\item Levantamento de requisitos do sistema, bem como das tecnologias que serão usadas no desenvolvimento.
		\item Desenvolver protótipos de maneira simplificada para se obter um conhecimento parcial das tecnologias em questão.
		\item Fazer um planejamento do projeto a partir dos requisitos coletados além de uma estimativa do tempo que será necessário.
		\item Desenvolver um aplicativo que permita aos alunos responderem o questionário da CPA e receber notificações quando o mesmo for liberado até seu encerramento.
		\item Realizar testes de funcionamento e estabilidade.
	\end{itemize}
	\par Tendo em mente esses objetivos, espera-se demonstrar de maneira simples e efetiva o funcionamento do Ionic e seu método de desenvolvimento híbrido, tirando o maior proveito possível com um software que auxiliará a pesquisa que é realizada CPA.

\chapter{JUSTIFICATIVAS}

	\par A escolha de falarmos sobre o Ionic e desenvolver um aplicativo usando seus conceitos, deve-se pela sua excelente forma de construir um App\footnote{Abreviação para a palavra \textit{Application}} \textit{mobile} multiplataforma, além das tecnologias usadas por ele que estão mais fortes do que nunca e em constantes atualizações. A principal plataforma utilizada será o \textit{Android} devido à sua grande popularidade e por ser um dos SO\footnote{Abreviação para Sistema Operacional} mais usados mundialmente.
	
	\par A relevância do aplicativo frente à universidade é grande, pois facilitará na pesquisa que é feita pela CPA, onde muitos alunos deixam de responder por falta da praticidade, que pode ser alcançada a partir de um aplicativo destinado especialmente para ela, com fácil utilização, intuitivo e que emite alertas aos usuários quando a pesquisa estiver disponível.
	
	\par O trabalho também auxiliará os alunos que se interessarem por aprender como funciona o Ionic e seu método de desenvolvimento, além de todas as tecnologias e \textit{frameworks} que estão envolvidos no processo.
	