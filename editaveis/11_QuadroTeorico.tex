\chapter{QUADRO TEÓRICO}

	\par Para que se possa desenvolver uma aplicação são necessárias algumas tecnologias como linguagens de programação e \textit{frameworks}, falaremos no capítulo a seguir quais são eles e suas principais características.

\section{Ionic}
	\par Segundo \citeonline{ionicdesc} o \textit{framework} foi construído por Ben Sperry, Adam Bradley e Max Lynch em Drifty, uma empresa de \textit{software} independente e que possui diversos produtos reconhecidos no mercado. O Ionic foi desenvolvido para possibilitar que aplicativos \textit{mobile} fossem desenvolvidos em sua base com HTML5. Mesmo com conhecimentos não muito aprofundados em tecnologias web, é possível criar um aplicativo usando Ionic, que é um grande aglomerado de tecnologias que trabalham juntas para fazer do desenvolvimento web um caminho estreito para o mundo \textit{mobile}.
	
	\par Existem três principais modos de construção de um aplicativo móvel, são eles: aplicativos nativos, sites móveis e aplicativos híbridos, onde cada um apresenta diversas vantagens e desvantagens, o Ionic se encaixa na terceira opção. \citeonline{ionicinaction} afirma que o desenvolvimento de aplicativos móveis se tornou um requisito inicial para qualquer desenvolvedor. O Ionic facilita esse desenvolvimento, ele faz com que um aplicativo móvel híbrido seja como um aplicativo nativo, através de recursos próprios que interpretam o código por partes, nas quais a interface gráfica é controlada por uma espécie de navegador isolado chamado de \textit{WebView}, e também um aplicativo nativo que controla o \textit{hardware}. Além do AngularJS, o Ionic possui o Cordova, que funciona carregando todo o código da aplicação web para apresentá-lo ao usuário, além de realizar a comunicação com o dispositivo.
	
	\par O uso de AngularJS atualmente é exigido pelo Ionic, isso a fim de potencializar seu desenvolvimento. Além de AngularJS, o Ionic também usa Javascript e possui CSS nativo.
	
	\subsection{Cordova}
		\par Segundo \citeonline{cordovadesc} o Apache Cordova é um \textit{framework} de código aberto para desenvolvimento móvel, com ele os aplicativos são direcionados para cada plataforma específica, contando com ligações API compatíveis para realizar o acesso ao \textit{hardware} de cada dispositivo.

	\subsection{CSS}
		\par Segundo \citeonline{silva2011css3}, CSS é a abreviação para o termo em inglês \textit{Cascading Style Sheet}, traduzido para o
		português como folhas de estilo em cascata. A linguagem CSS tem por finalidade estilizar uma estrutura HTML para a apresentação de elementos. Como por exemplo: cores de fontes, tamanhos de texto, bordas arredondadas entre outros. De acordo com \citeonline{lie2005cascading} os primeiros vestígios de CSS são do ano de 1994, onde surgiu a necessidade de melhorar a aparência de uma página web tendo como \textit{layout} um jornal, naquela época algumas coisas ainda eram impossíveis de se fazer, coisas que hoje em dia com o CSS3, encontramos facilmente em qualquer página web.
		\par Sendo assim, entendemos que o CSS faz todo o trabalho de estilizar e deixar páginas web com uma interface amigável e melhor apresentada do que somente quando se tem HTML puro.
		
	\subsection{Javascript}
		\par Segundo \citeonline{silva2010javascript}, o Javascript foi criado pela Netscape em conjunto com a Sun Microsystems com o objetivo de adicionar interatividade a páginas web que por padrão são estáticas. Com HTML puro não é possível processar dados ou enviá-los ao servidor, o Javascript foi desenvolvido para rodar no lado do cliente (\textit{client-side}) assim, apenas um navegador é necessário para fazermos funcionar um código escrito em Javascript.
		\par De acordo com \citeonline{flanagan2006javascript}, o Javascript é interpretado ao invés de compilado, por isso muitas vezes é considerado uma linguagem de \textit{script} ao invés de linguagem de programação. Apesar de levar parte do nome, o Javascript não é uma versão simplificada do Java, que é uma linguagem desenvolvida também pela Sun Microsystems.
		
\section{AngularJS}
	\par Segundo \citeonline{angularjsdoc}, o AngularJS é um \textit{framework} para desenvolvimento de aplicações web, foi criado em 2009 por Misko Hevery e Adams Abrons, porém adotado posteriormente pelo Google.
	\par \citeonline{branas2014angularjs} diz que devido à sua forma de escrita simples e direta, reutilizável e que possibilita uma aplicação sustentável, a escrita do código com Angular ao seu final elimina uma enorme quantidade de código desnecessário, assim uma equipe de desenvolvimento mantém seu foco no trabalho que é realmente importante.
	\par De acordo com \citeonline{green2013angularjs}, com a evolução das tecnologias web, as aplicações foram se tornando maiores e mais robustas, isso acarretou em um aumento da complexidade para os seus administradores. Desenvolver utilizando Javascript/JQuery não estava gerando desempenho suficiente e a manutenção de código à longo prazo estava prejudicada. O AngularJS surgiu para atender essas necessidades que surgiram, com ele muitas tecnologias acabaram por ser dispensáveis ou pouco utilizadas.
	\par O AngularJS utiliza como modelo de arquitetura o padrão MVC, que divide o sistema em três partes distintas e modulares: o modelo (\textit{model}), a visão (\textit{view}) e o controlador (\textit{controller}). O modelo corresponde aos dados, visão é a interface gráfica do aplicativo com o usuário e o controlador corresponde à lógica de negócios. Cada uma dessas partes separa o código em grupos, o que proporciona vantagens como o dimensionamento e a organização de tarefas, onde cada uma faz somente o que é designado, além disso o código se torna reutilizável e de fácil manutenção. 
	

%\section{PHP}

%\section{MySQL}

%\section{HTML5}