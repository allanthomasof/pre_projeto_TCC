\chapter{QUADRO TEÓRICO}

	\par Para que se possa desenvolver uma aplicação são necessárias algumas tecnologias como linguagens de programação e \textit{frameworks}, falaremos no capítulo a seguir quais são eles e suas principais características.

\section{IONIC}
	\par Segundo \citeonline{ionicdesc} o \textit{framework} foi construído por benjsperry , adamdbradley e maxlynch em Drifty , uma empresa de \textit{software} independente e fabricantes de produtos reconhecidos como \textit{Codiqa} e \textit{Jetstrap}. O IONIC foi desenvolvido para possibilitar que aplicativos \textit{mobile} fossem desenvolvidos em sua base com HTML5.
	\par Atualmente o IONIC exige o uso de AngularJS, que será o próximo assunto, isso para fins de potencializar seu desenvolvimento. Além do AngularJS, o IONIC também usa \textit{Javascript} e CSS nativo.

	\subsection{CSS}
		\par Segundo \citeonline{silva2011css3} CSS é a abreviação para o termo em inglês \textit{Cascading Style Sheet}, traduzido para o
		português como folhas de estilo em cascata. A linguagem CSS tem por finalidade estilizar uma estrutura HTML para a apresentação de elementos. Como por exemplo: cores de fontes, tamanhos de texto, bordas arredondadas entre outros.
		\par De acordo com \citeonline{lie2005cascading} os primeiros vestígios de CSS são do ano de 1994, onde surgiu a necessidade de melhorar a aparência de uma página \textit{web} tendo como \textit{layout} um jornal, naquela época algumas coisas ainda eram impossíveis de se fazer, coisas que hoje em dia com o CSS3, encontramos facilmente em qualquer \textit{website}.
		\par Sendo assim, entendemos que o CSS faz todo o trabalho de estilizar e deixar páginas \textit{web} com uma interface amigável e melhor apresentada do que somente quando se tem HTML puro.
		
\section{AngularJS}
	\par AngularJS
	
\section{Cordova}
	\par Cordova

\section{PHP}
	\par PHP

\section{MySQL}
	\par MySQL

\section{HTML5}
	\par HTML5

\section{CSS}
	\par CSS