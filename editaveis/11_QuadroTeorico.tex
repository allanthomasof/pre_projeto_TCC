\chapter{QUADRO TEÓRICO}

	\par Para que se possa desenvolver uma aplicação são necessárias algumas tecnologias como linguagens de programação e \textit{frameworks}, falaremos no capítulo a seguir quais são eles e suas principais características.

\section{IONIC}
	\par Segundo \citeonline{ionicdesc} o \textit{framework} foi construído por benjsperry, adamdbradley e maxlynch em Drifty, uma empresa de \textit{software} independente e fabricantes de produtos reconhecidos como \textit{Codiqa} e \textit{Jetstrap}. O IONIC foi desenvolvido para possibilitar que aplicativos \textit{mobile} fossem desenvolvidos em sua base com HTML5.
	\par \citeonline{ionicinaction} afirma que desenvolvimento de aplicativos móveis se tornou um um requisito inicial para qualquer desenvolvedor, o IONIC facilita esse desenvolvimento, ele faz com que um aplicativo móvel híbrido seja como um app nativo, porém com uma enorme facilidade na criação da sua interface gráfica. Além do AngularJS, o IONIC possui o Cordova, que funciona carregando todo o código da aplicação web para apresentá-lo ao usuário, além de realizar a comunicação com o dispositivo(\textit{hardware}).
	\par O uso de AngularJS(tema da sessão seguinte) atualmente é exigido pelo IONIC, isso para fins de potencializar seu desenvolvimento. Além de AngularJS, o IONIC também usa Javascript e possui CSS nativo.
	
	\subsection{Cordova}
		\par Segundo \citeonline{cordovadesc} o Apache Cordova é um \textit{framework} de código aberto para desenvolvimento móvel, com ele os aplicativos são direcionados para cada plataforma específica, contando com ligações API compatíveis para realizar o acesso ao \textit{hardware} de cada dispositivo.

	\subsection{CSS}
		\par Segundo \citeonline{silva2011css3} CSS é a abreviação para o termo em inglês \textit{Cascading Style Sheet}, traduzido para o
		português como folhas de estilo em cascata. A linguagem CSS tem por finalidade estilizar uma estrutura HTML para a apresentação de elementos. Como por exemplo: cores de fontes, tamanhos de texto, bordas arredondadas entre outros.
		\par De acordo com \citeonline{lie2005cascading} os primeiros vestígios de CSS são do ano de 1994, onde surgiu a necessidade de melhorar a aparência de uma página \textit{web} tendo como \textit{layout} um jornal, naquela época algumas coisas ainda eram impossíveis de se fazer, coisas que hoje em dia com o CSS3, encontramos facilmente em qualquer \textit{website}.
		\par Sendo assim, entendemos que o CSS faz todo o trabalho de estilizar e deixar páginas \textit{web} com uma interface amigável e melhor apresentada do que somente quando se tem HTML puro.
		
	\subsection{Javascript}
		\par Segundo \citeonline{silva2010javascript} o Javascript foi criado pela Netscape em conjunto com a Sun Microsystems com o objetivo de adicionar interatividade a páginas \textit{web} que por padrão são estáticas. Com HTML puro não é possível processar dados ou enviá-los ao servidor, sendo que o Javascript foi desenvolvido para rodar no lado do cliente(\textit{client-side}) assim, apenas um navegador é necessário para fazermos funcionar um código escrito em Javascript.
		\par De acordo com \citeonline{flanagan2006javascript} javascript é interpretado ao invés de compilado, por isso muitas vezes é considerado uma linguagem de \textit{script} ao invés de linguagem de programação. Apesar de levar parte do nome, o Javascript não é uma versão simplificada do Java, que é uma linguagem desenvolvida também pela Sun Microsystems.
		
\section{AngularJS}
	\par AngularJS

\section{PHP}
	\par PHP

\section{MySQL}
	\par MySQL

\section{HTML5}
	\par HTML5