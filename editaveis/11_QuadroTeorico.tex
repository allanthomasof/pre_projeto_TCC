\chapter{QUADRO TEÓRICO}

	\par Para que se possa desenvolver uma aplicação são necessárias algumas tecnologias como linguagens de programação e \textit{frameworks}, falaremos no capítulo a seguir quais são eles e suas principais características.

\section{Ionic}
	\par Segundo \citeonline{ionicdesc} o \textit{framework} foi construído por Ben Sperry, Adam Bradley e Max Lynch em Drifty, uma empresa de \textit{software} independente e que possui diversos produtos reconhecidos no mercado. O Ionic foi desenvolvido para possibilitar que aplicativos \textit{mobile} fossem desenvolvidos em sua base com HTML5. Mesmo com conhecimentos não muito aprofundados em tecnologias web, é possível criar um aplicativo usando Ionic, que é um grande aglomerado de tecnologias que trabalham juntas para fazer do desenvolvimento web um caminho estreito para o mundo \textit{mobile}.
	
	\par Existem três principais modos de construção de um aplicativo móvel, são eles: aplicativos nativos, sites móveis e aplicativos híbridos, onde cada um apresenta diversas vantagens e desvantagens, o Ionic se encaixa na terceira opção. \citeonline{ionicinaction} afirma que o desenvolvimento de aplicativos móveis se tornou um requisito inicial para qualquer desenvolvedor. O Ionic facilita esse desenvolvimento, ele faz com que um aplicativo móvel híbrido seja como um aplicativo nativo, através de recursos próprios que interpretam o código por partes, nas quais a interface gráfica é controlada por uma espécie de navegador isolado chamado de \textit{WebView}, e também um aplicativo nativo que controla o \textit{hardware}. Além do AngularJS, o Ionic possui o Cordova, que funciona carregando todo o código da aplicação web para apresentá-lo ao usuário, além de realizar a comunicação com o dispositivo.
	
	\par O uso de AngularJS atualmente é exigido pelo Ionic, isso a fim de potencializar seu desenvolvimento. Além de AngularJS, o Ionic também usa Javascript e possui CSS nativo.
	
	\subsection{Cordova}
		\par Segundo \citeonline{cordovadesc} o Apache Cordova é um \textit{framework} de código aberto para desenvolvimento móvel, com ele os aplicativos são direcionados para cada plataforma específica, contando com ligações API compatíveis para realizar o acesso ao \textit{hardware} de cada dispositivo.

	\subsection{CSS}
		\par Segundo \citeonline{silva2011css3}, CSS é a abreviação para o termo em inglês \textit{Cascading Style Sheet}, traduzido para o
		português como folhas de estilo em cascata. A linguagem CSS tem por finalidade estilizar uma estrutura HTML para a apresentação de elementos. Como por exemplo: cores de fontes, tamanhos de texto, bordas arredondadas entre outros. De acordo com \citeonline{lie2005cascading} os primeiros vestígios de CSS são do ano de 1994, onde surgiu a necessidade de melhorar a aparência de uma página web tendo como \textit{layout} um jornal, naquela época algumas coisas ainda eram impossíveis de se fazer, coisas que hoje em dia com o CSS3, encontramos facilmente em qualquer página web.
		\par Sendo assim, entendemos que o CSS faz todo o trabalho de estilizar e deixar páginas web com uma interface amigável e melhor apresentada do que somente quando se tem HTML puro.
		
	\subsection{Javascript}
		\par Segundo \citeonline{silva2010javascript}, o Javascript foi criado pela Netscape em conjunto com a Sun Microsystems com o objetivo de adicionar interatividade a páginas web que por padrão são estáticas. Com HTML puro não é possível processar dados ou enviá-los ao servidor, o Javascript foi desenvolvido para rodar no lado do cliente (\textit{client-side}) assim, apenas um navegador é necessário para fazermos funcionar um código escrito em Javascript.
		\par De acordo com \citeonline{flanagan2006javascript}, o Javascript é interpretado ao invés de compilado, por isso muitas vezes é considerado uma linguagem de \textit{script} ao invés de linguagem de programação. Apesar de levar parte do nome, o Javascript não é uma versão simplificada do Java, que é uma linguagem desenvolvida também pela Sun Microsystems.
		
\section{AngularJS}
	\par Segundo \citeonline{angularjsdoc}, o AngularJS é um \textit{framework} para desenvolvimento de aplicações web, foi criado em 2009 por Misko Hevery e Adams Abrons, porém adotado posteriormente pelo Google.
	\par \citeonline{branas2014angularjs} diz que devido à sua forma de escrita simples e direta, reutilizável e que possibilita uma aplicação sustentável, a escrita do código com Angular ao seu final elimina uma enorme quantidade de código desnecessário, assim uma equipe de desenvolvimento mantém seu foco no trabalho que é realmente importante.
	\par De acordo com \citeonline{green2013angularjs}, com a evolução das tecnologias web, as aplicações foram se tornando maiores e mais robustas, isso acarretou em um aumento da complexidade para os seus administradores. Desenvolver utilizando Javascript/JQuery não estava gerando desempenho suficiente e a manutenção de código à longo prazo estava prejudicada. O AngularJS surgiu para atender essas necessidades que surgiram, com ele muitas tecnologias acabaram por ser dispensáveis ou pouco utilizadas.
	\par O AngularJS utiliza como modelo de arquitetura o padrão MVC, que divide o sistema em três partes distintas e modulares: o modelo (\textit{model}), a visão (\textit{view}) e o controlador (\textit{controller}). O modelo corresponde aos dados, visão é a interface gráfica do aplicativo com o usuário e o controlador corresponde à lógica de negócios. Cada uma dessas partes separa o código em grupos, o que proporciona vantagens como o dimensionamento e a organização de tarefas, onde cada uma faz somente o que é designado, além disso o código se torna reutilizável e de fácil manutenção. 
	
\section{PHP}
	\par O PHP\footnote{\textit{Personal Home Page}} é uma linguagem \textit{open source}, que em tradução livre, significa código aberto, ou seja, qualquer um pode contribuir para seus aprimoramentos. De acordo com \citeonline{phpdoc}, o que o difere do Javascript, por exemplo, é que o PHP é executado no lado do servidor, o que faz com que o código seja processado por um sistema dedicado especialmente para essa função, e não na máquina do usuário. Desta forma, o navegador recebe apenas os resultados do processamento, o que é indispensável quando se é necessário ter sigilo e segurança com as informações.
	\par Segundo \citeonline{niederauer2004desenvolvendo}, o PHP é uma das linguagens de programação mais conhecidas e utilizadas mundialmente. Sua função, é principalmente trazer interações e conteúdos dinâmicos para páginas web, em que apenas o HTML não é suficiente. Um site de notícias, por exemplo, só possui atualizações em tempo real por causa do PHP ou uma linguagem semelhante, com a função de recuperar informações de um banco de dados e transmiti-las para o usuário.
	\par Para que seja possível a comunicação entre uma aplicação móvel e um servidor de banco de dados através do PHP, é necessário um serviço web realizando esse intermédio. Para essa função será utilizado o \textit{Slim Framework}, segundo \citeonline{slimdoc}, o \textit{Slim Framework} é uma ferramenta que possibilita a criação de APIs\footnote{\textit{Application Programming Interface}} REST\footnote{\textit{Representational State Transfer}} de maneira sucinta e poderosa através do protocolo HTTP\footnote{\textit{Hypertext Transfer Protocol}}.
	
\section{REST}
	\par Segundo \citeonline{saudate2014rest}, o REST criado por Roy Fielding, nada mais é que um estilo de desenvolvimento de \textit{web services}, em que a tecnologia trabalha com base no protocolo HTTP e seus respectivos métodos, realizando chamadas através de URL's\footnote{\textit{Uniform Resource Locator}}. 
	\par A URL deve ser única por recurso, ou seja, para cada função que se deseja ter, é necessário uma URL diferente, pois é ela quem define o procedimento que será feito e qual será a interação com o lado do cliente. Em conjunto com a URL, é necessário trabalhar com os métodos HTTP, pois são eles quem a API interpreta para saber o que deverá ser feito. Na versão corrente do HTTP, 1.1, estão disponíveis oito principais métodos, são eles:
	\begin{itemize}
	 	\item GET
	 	\item POST
	 	\item PUT
	 	\item DELETE
	 	\item OPTIONS
	 	\item HEAD
	 	\item TRACE
	   	\item CONNECT
	\end{itemize}
	\par REST trabalha com dois principais formatos de envio e retorno de informações, que são o XML\footnote{\textit{eXtensible Markup Language}} e o JSON\footnote{\textit{JavaScript Object Notation}}. Ambos os formatos possuem a mesma proposta, porém, com algumas diferenças, principalmente na implementação. No projeto será usado o formato JSON, por conta de sua performance, menor complexidade, tamanho reduzido quando comparado com o XML, e também por ser muito usado atualmente.
	\par A grande vantagem do uso das REST API's é a possibilidade de comunicação de dados entre domínios distintos, justamente a situação de uma aplicação \textit{mobile}, diferente do que acontece nos websites por exemplo, onde ambos estão na mesma rede e se comunicam diretamente sem a necessidade de um \textit{web service} realizando o intermédio.

\section{MySQL}
	\par Segundo \citeonline{mysql2010jobstraibizer}, o MySQL foi criado na Suécia por três desenvolvedores, Allan Larsson, David Amark e Michael Widenius, o sistema foi posteriormente adquirido pela Sun Microsystems, empresa que hoje pertence à Oracle.
	\par \citeonline{niederauer2005integrando} diz que o MySQL é um SGBD\footnote{Sistema de Gerenciamento de Bancos de Dados} relacional e utiliza linguagem SQL\footnote{\textit{Structured Query Language}}. O MySQL é muito utilizado em aplicações web e sempre disponibilizado por empresas de hospedagem, está entre os principais bancos de dados de código aberto. Sua alta popularidade está relacionada à disponibilidade para diferentes sistemas operacionais como o Linux, Mac e Windows. Entre suas vantagens estão a performance, escalabilidade e confiabilidade, além de ser compatível com diversas linguagens de programação como o PHP, Java, Python e C++. O MySQL é um banco de dados excelente, tanto para pequenos websites como para grandes portais e aplicações web.



%\section{HTML5}